\documentclass[11pt, a4paper, twoside]{article}   	% use "amsart" instead of "article" for AMSLaTeX format

\usepackage{geometry}                		% See geometry.pdf to learn the layout options. There are lots.
\usepackage{pdfpages}
\usepackage{caption}
\usepackage{minted}
\usepackage[german]{babel}			% this end the next are needed for german umlaute
\usepackage[utf8]{inputenc}
\usepackage{color}
\usepackage{graphicx}
\usepackage{titlesec}
\usepackage{fancyhdr}
\usepackage{lastpage}
\usepackage{hyperref}
% http://www.artofproblemsolving.com/wiki/index.php/LaTeX:Symbols#Operators
% =============================================
% Layout & Colors
% =============================================
\geometry{
   a4paper,
   total={210mm,297mm},
   left=20mm,
   right=20mm,
   top=20mm,
   bottom=30mm
 }	

\definecolor{myred}{rgb}{0.7,0.3,0.4}
\definecolor{mygreen}{rgb}{0,0.6,0}
\definecolor{mygray}{rgb}{0.5,0.5,0.5}
\definecolor{mymauve}{rgb}{0.58,0,0.82}

\setcounter{secnumdepth}{4}

% the default java directory structure and the main packages
\newcommand{\appRoot}{../source/AmazingRace/app}
\newcommand{\javaRoot}{\appRoot/src/main/java/at/fh/ooe/moc5/amazingrace}
\newcommand{\resRoot}{\appRoot/src/main/res}
\newcommand{\activityRoot}{\javaRoot/activity}
\newcommand{\adapterRoot}{\javaRoot/adapter}
\newcommand{\jsonModelRoot}{\javaRoot/model/json}
\newcommand{\taskModelRoot}{\javaRoot/model/task}
\newcommand{\viewModelRoot}{\javaRoot/model/view}
\newcommand{\serviceRoot}{\javaRoot/service}
\newcommand{\utilRoot}{\javaRoot/util}
\newcommand{\watcherRoot}{\javaRoot/watcher}
\newcommand{\layoutRoot}{\resRoot/layout}
\newcommand{\valuesRoot}{\resRoot/values}
\newcommand{\manifestsRoot}{\appRoot/src/main}
\newcommand{\imagesRoot}{images}

% =============================================
% Code Settings
% =============================================
\newenvironment{code}{\captionsetup{type=listing}}{}
\newmintedfile[javaSourceFile]{java}{
	linenos=true, 
	frame=single, 
	breaklines=true, 
	tabsize=2,
	numbersep=5pt,
	xleftmargin=10pt,
	baselinestretch=1,
	fontsize=\footnotesize
}
\newmintinline[inlineJava]{java}{}
\newminted[javaSource]{java}{
	breaklines=true, 
	tabsize=2,
	autogobble=true,
	breakautoindent=false
}
\newmintedfile[xmlSourceFile]{xml}{
	linenos=true, 
	frame=single, 
	breaklines=true, 
	tabsize=2,
	numbersep=5pt,
	xleftmargin=10pt,
	baselinestretch=1,
	fontsize=\footnotesize
}
\newmintedfile[propertiesFile]{properties}{
	linenos=true, 
	frame=single, 
	breaklines=true, 
	tabsize=2,
	numbersep=5pt,
	xleftmargin=10pt,
	baselinestretch=1,
	fontsize=\footnotesize
}
% =============================================
% Page Style, Footers & Headers, Title
% =============================================
\title{Übung 3}
\author{Thomas Herzog}

\lhead{Übung 3}
\chead{}
\rhead{\includegraphics[scale=0.10]{FHO_Logo_Students.jpg}}

\lfoot{S1310307011}
\cfoot{}
\rfoot{ \thepage / \pageref{LastPage} }
\renewcommand{\footrulewidth}{0.4pt}
% =============================================
% D O C U M E N T     C O N T E N T
% =============================================
\pagestyle{fancy}
\begin{document}
\setlength{\headheight}{15mm}
%\includepdf[pages={1-3}]{Swe4xA05-BB.pdf}
{\color{myred}
	\section
		{AmazingRace Android Anwendung}
}
Folgende Dokumentation dokumentiert die Entwicklung der mobilen Anwendung \emph{AmazingRace}, die in \emph{Java 7} für die Android-Plattform \emph{API-Level 22, Android 5.1.1} entwickelt wurde.
\newline

\subsection{Lösungsidee}
Die Anwendung soll in auf drei Aktivitäten aufgeteilt werden, wobei die Login-Aktivität als Launcher-Aktivität definiert werden soll. Über diese Aktivität wird die Anwendung gestartet. Nach einem erfolgreichem Login soll diese Aktivität beendet werden, da es keinen Grund gibt warum ein Benutzer wieder auf diese Aktivität zurückkehren sollte.
\newline
\newline
Nach dem Login wird der Benutzer zu einer Übersicht aller Routen weitergeleitet, wo er eine Route auswählen kann. In dieser Ansicht sollen die einzelnen Routen mit einem Kontext-Menu versehen werden, die je nach Zustand der Route folgende Optionen bereitstellen soll.
\begin{enumerate}
	\item\emph{Play (Es gibt einen offenen Checkpoint)}
	\newline
	Wechselt zur Checkpoint-Aktivität.
	\item\emph{Open (Route bereits beendet)}
	\newline
	Wechselt zur Checkpoint-Aktivität
	\item\emph{Reset (Wenn zumindest ein Checkpoint besucht wurde)}
	\newline
	Setzt diese Route zurück und verbleibt auf dieser Aktivität	
\end{enumerate}
\ \newline
Ebenfalls soll eine Action-Bar implementiert werden, die folgende Optionen bereitstellten soll.
\begin{enumerate}
	\item\emph{Reset}
	\newline
	Setzt alle Routen zurück
	\item\emph{Reload}
	\newline
	Lädt alle Routen neu.
	\item\emph{Close}
	\newline
	Öffnet einen Dialog der im Falle einer Bestätigung die Anwendung beendet
\end{enumerate} 
\ \newline
Für alle Ladeoperationen soll ein \emph{ProgressDialog} angezeigt werden, der eine dementsprechende Meldung beinhaltet (Z.b.: Load routes ...). Der verwendete Service soll über ein Interface abgebildet werden, somit beinhaltet die benutzenden Klasse keine Referenzen auf die konkrete Implementierung. Die Implementierung soll über eine \emph{Factory} bereitgestellt werden, die die Service Instanz als \emph{Singleton} verwaltet, da hier nur eine Instanz benötigt wird. Des Weiteren sollen \emph{View-Models} implementiert werden, die die gesamte View-Logik beinhalteten und keine Referenzen auf Android-Ressourcen beinhalten sollen.
\newline
\newline
Die Aktivitäten sollen die gesamte Interaktionslogik beinhalten und die möglichen \emph{ServiceExceptions} der Serviceaufrufe, die über die View-Model erfolgen sollen und von diesen weitergereicht werden, behandeln, da hierbei Referenzen auf Android-Ressourcen benötigt werden. Diese \emph{ServiceException} soll in einer abstrakten Basisklasse behandelt werden, von der alle Aktivitäten ableiten sollen. Somit ist diese Fehlerbehandlung (Anzeige für den Benutzer) in einer Klasse gekapselt und trotzdem können die Aktivitäten ihrerseits Fehlerbehandlungen durchführen.
\newpage

\subsubsection{Architektur}
Folgend ist die Architektur der Anwedung \emph{AmazingRace} dokumentiert.
\newline
\newline
Folgende Abbildung zeigt die Hierarchie der Klasse \emph{AbstractActivity} die als Basisklasse aller Aktivitäten fungiert und gemeinsam genutzte Ressourcen und Konstanten bereitstellt.
\begin{figure}[h]
	\centering
	\includegraphics[scale=0.4]{\imagesRoot/abstract_activity_hierarchy.PNG}
	\caption
	{Klassenhierarchie der Klasse \emph{AbstractActivity}}
\end{figure}
\ \newline
Folgende Abbildung zeigt die Paketstruktur der Anwendung \emph{AmazingRace}. Als Wurzelnamensraum wurde \emph{at.fh.ooe.moc5.amazingrace} gewählt unter dem sich alle Ressourcen befinden.
\begin{figure}[h]
	\centering
	\includegraphics[scale=0.65]{\imagesRoot/amazingrace_package_structure.PNG}
	\caption
	{Paketstruktur der Anwendung}
\end{figure}
\ \newline
\newpage

Folgende Abbildung zeigt die Ressourcenverzeichnisse in denen alle Ressourcen wie \emph{Layout} usw. liegen. Es wurde hierbei darauf verzichtet, die Ressourcen für mehrere Auflösungen und Sprachen zu definieren.
\begin{figure}[h]
	\centering
	\includegraphics[scale=0.65]{\imagesRoot/amazingrace_resource_structure.PNG}
	\caption
	{Ressourcen der Anwendung}
\end{figure}
\ \newline
Des Weiteren wurde eine Klasse \emph{AmazingRaceApplication} eingeführt, die die nicht abgefangene Exceptions behandelt, den Benutzerkontext hält und gemeinsam verwendete Konstanten definiert.
\newline
\subsubsection{Setup}
Da ein Google Android API-Key verwendet wird wurde ein eigenes Zertifikat erstellt welches bei Google-Developer-Console registriert wurde. Dieses Zertifikat muss im Android-Studio bei der \emph{Debug} und \emph{Release} Konfiguration eingestellt werden, da ansonsten Google Maps nicht funktionieren wird. Es sollte zwar ein Zertifikat für die \emph{Debug} und \emph{Release} Konfiguration erstellt werden, aber es wurde hierbei darauf verzichtet und es wird nur ein Zertifikat für beide Konfigurationen verwendet.
\begin{enumerate}
	\item\emph{Alias:} amazingrace
	\item\emph{Key-Password:} amazingrace
	\item\emph{Keystore-Password:} amazingrace
\end{enumerate}
\ \newline
Der Keystore befindet sich im Android-Studio-Projekt Wurzelverzeichnis und trägt den Name \emph{amazingrace.jks}
\newline
\newline
Des Weiteren wurde folgender Emulator verwendet
\begin{enumerate}
	\item\emph{Device:} Nexus 5
	\item\emph{API-Level:} 22
	\item\emph{Architektur:} x86\_64 (schneller und \emph{ARM}  hatte falsche Play-Service Version)
\end{enumerate}
\ \newline
Die Applikation wurde ebenfalls auf eine \emph{Samsung S6-Edge} getestet.
\newpage

\subsection{Source}
Im folgenden ist der gesamte Source der Anwendung aufgeführt.

\subsubsection{Manifest}
\begin{code}
	\caption{AndroidManifest.xml}
	\xmlSourceFile{\manifestsRoot/AndroidManifest.xml}
\end{code}
\newpage
\subsubsection{Aktivitäten}
\begin{code}
	\caption{AmazingRaceApplication.java}
	\javaSourceFile{\javaRoot/AmazingRaceApplication.java}
\end{code}
\newpage
\begin{code}
	\caption{AbstractActivity.java}
	\javaSourceFile{\activityRoot/AbstractActivity.java}
\end{code}
\newpage
\begin{code}
	\caption{LoginActivity.java}
	\javaSourceFile{\activityRoot/LoginActivity.java}
\end{code}
\newpage
\begin{code}
	\caption{RouteActivity.java}
	\javaSourceFile{\activityRoot/RouteActivity.java}
\end{code}
\newpage
\begin{code}
	\caption{CheckpointActivity.java}
	\javaSourceFile{\activityRoot/CheckpointActivity.java}
\end{code}
\newpage

\subsubsection{Adapter}
\begin{code}
	\caption{RouteArrayAdapter.java}
	\javaSourceFile{\adapterRoot/RouteArrayAdapter.java}
\end{code}
\newpage

\subsubsection{JSON-Models}
\begin{code}
	\caption{CheckpointModel.java}
	\javaSourceFile{\jsonModelRoot/CheckpointModel.java}
\end{code}
\newpage
\begin{code}
	\caption{CheckpointRequestModel.java}
	\javaSourceFile{\jsonModelRoot/CheckpointRequestModel.java}
\end{code}
\newpage
\begin{code}
	\caption{CredentialsRequestModel.java}
	\javaSourceFile{\jsonModelRoot/CredentialsRequestModel.java}
\end{code}
\newpage
\begin{code}
	\caption{RouteModel.java}
	\javaSourceFile{\jsonModelRoot/RouteModel.java}
\end{code}
\newpage
\begin{code}
	\caption{RouteRequestModel.java}
	\javaSourceFile{\jsonModelRoot/RouteRequestModel.java}
\end{code}
\newpage

\subsubsection{Task-Models}
\begin{code}
	\caption{AsyncTaskResult.java}
	\javaSourceFile{\taskModelRoot/AsyncTaskResult.java}
\end{code}
\newpage

\subsubsection{View-Models}
\begin{code}
	\caption{LoginViewModel.java}
	\javaSourceFile{\viewModelRoot/LoginViewModel.java}
\end{code}
\newpage
\begin{code}
	\caption{RoutesViewModel.java}
	\javaSourceFile{\viewModelRoot/RoutesViewModel.java}
\end{code}
\newpage
\begin{code}
	\caption{UserContextModel.java}
	\javaSourceFile{\viewModelRoot/UserContextModel.java}
\end{code}
\newpage
\begin{code}
	\caption{CheckpointViewModel.java}
	\javaSourceFile{\viewModelRoot/CheckpointViewModel.java}
\end{code}
\newpage

\subsubsection{Service}
\begin{code}
	\caption{ServiceProxy.java}
	\javaSourceFile{\serviceRoot/ServiceProxy.java}
\end{code}
\newpage
\begin{code}
	\caption{RestServiceProxyImpl.java}
	\javaSourceFile{\serviceRoot/RestServiceProxyImpl.java}
\end{code}
\newpage
\begin{code}
	\caption{ServiceException.java}
	\javaSourceFile{\serviceRoot/ServiceException.java}
\end{code}
\newpage
\begin{code}
	\caption{ServiceProxyFactory.java}
	\javaSourceFile{\serviceRoot/ServiceProxyFactory.java}
\end{code}
\newpage

\subsubsection{Utilities}
\begin{code}
	\caption{DialogUtil.java}
	\javaSourceFile{\utilRoot/DialogUtil.java}
\end{code}
\newpage

\subsubsection{Watcher}
\begin{code}
	\caption{AnswerButtonTextWatcher.java}
	\javaSourceFile{\watcherRoot/AnswerButtonTextWatcher.java}
\end{code}
\newpage
\begin{code}
	\caption{CheckpointViewModelBindingTextWatcher.java}
	\javaSourceFile{\watcherRoot/CheckpointViewModelBindingTextWatcher.java}
\end{code}
\newpage
\begin{code}
	\caption{LoginButtonTextWatcher.java}
	\javaSourceFile{\watcherRoot/LoginButtonTextWatcher.java}
\end{code}
\newpage
\begin{code}
	\caption{LoginViewModelBindingTextWatcher.java}
	\javaSourceFile{\watcherRoot/LoginViewModelBindingTextWatcher.java}
\end{code}
\newpage

\subsection{Layout}
\begin{code}
	\caption{activity\_login.xml}
	\xmlSourceFile{\layoutRoot/activity_login.xml}
\end{code}
\newpage
\begin{code}
	\caption{activity\_route.xml}
	\xmlSourceFile{\layoutRoot/activity_route.xml}
\end{code}
\newpage
\begin{code}
	\caption{activity\_checkpoint.xml}
	\xmlSourceFile{\layoutRoot/activity_checkpoint.xml}
\end{code}
\begin{code}
	\caption{view\_checkpoint\_item.xml}
	\xmlSourceFile{\layoutRoot/view_checkpoint_item.xml}
\end{code}
\newpage
\begin{code}
	\caption{view\_route\_item.xml}
	\xmlSourceFile{\layoutRoot/view_route_item.xml}
\end{code}
\begin{code}
	\caption{view\_congratulations\_dialog.xml}
	\xmlSourceFile{\layoutRoot/view_congratulations_dialog.xml}
\end{code}
\newpage

\subsection{Values}
\begin{code}
	\caption{strings.xml}
	\xmlSourceFile{\valuesRoot/strings.xml}
\end{code}
\newpage

\subsection{Tests}
Folgend sind die Tests der Anwendung \emph{AmazingRace} angeführt.
\subsubsection{Login Aktivität}
Folgend sind die Tests der Login Aktivität angeführt.
\begin{figure}[h]
	\centering
	\includegraphics[scale=0.4]{\imagesRoot/activity_login.PNG}
	\caption
	{Login Aktivität}
\end{figure}
\ \newline
Der Button \emph{Login} ist hierbei solange deaktiviert solange eine der beiden Eingabefelder keinen Text beinhaltet. Dies bedeutet aber auch, das leere Passwörter nicht unterstützt werden.
\begin{figure}[h]
	\centering
	\includegraphics[scale=0.4]{\imagesRoot/activity_login_failed.PNG}
	\caption
	{Login fehlgeschlagen}
\end{figure}
\ \newline
Sollte der Login fehlschlagen so werden beide Eingabefelder zurückgesetzt und eine Fehlermeldung in Form eines \emph{Toasts} angezeigt.
\newpage
\begin{figure}[h]
	\centering
	\includegraphics[scale=0.4]{\imagesRoot/activity_login_no_network.PNG}
	\caption
	{Keine Internetverbindung}
\end{figure}
Sollte keine Netzwerkverbindung vorhanden sein so wird ein entsprechender \emph{Dialog} angezeigt.
\newline
\subsubsection{Routen Aktivität}
Folgend sind die Tests für die Routen Aktivität angeführt.
\begin{figure}[h]
	\centering
	\includegraphics[scale=0.4]{\imagesRoot/activity_routes.PNG}
	\caption
	{Routen Aktivität}
\end{figure}
\ \newline
Nach einem erfolgreichen Login oder durch Drücken des \emph{Back Button} in der Checkpoint Aktivität gelangt man zur Übersicht der verfügbaren Routen. Hierbei werden die Routen wie folgt sortiert.
\begin{enumerate}
	\item Offene Routen
	\begin{enumerate}
		\item 
	Anzahl der besuchten Checkpoints absteigend
		\item Routenname aufsteigend
	\end{enumerate}
	\item Abgeschlossene Routen
	\begin{enumerate}
		\item Routenname aufsteigend
	\end{enumerate}
\end{enumerate}
\ \newline
Ist eine Route bereits abgeschlossen so wird eine entsprechender Icon angezeigt.
\newpage
\begin{figure}[h]
	\centering
	\includegraphics[scale=0.4]{\imagesRoot/activity_routes_item_open_menu.PNG}
	\caption
	{Kontextmenü einer offenen Route}
\end{figure}
Ist eine Route offen so wird beim Kontextmenü lediglich eine Option \emph{Play} angezeigt.
\newline
\begin{figure}[h]
	\centering
	\includegraphics[scale=0.4]{\imagesRoot/activity_routes_item_progress_menu.PNG}
	\caption
	{Kontextmenü einer gestarteten Route}
\end{figure}
\ \newline
Wurde eine Route bereits gestartet so werden beim Kontextmenü zwei Optionen \emph{Play, Reset} angezeigt.
\newline
\begin{figure}[h]
	\centering
	\includegraphics[scale=0.4]{\imagesRoot/activity_routes_item_finished_menu.PNG}
	\caption
	{Kontextmenü einer abgeschlossenen Route}
\end{figure}
\ \newline
Ist eine Route bereits abgeschlossen so werden beim Kontextmenü zwei Optionen \emph{Open, Reset} angezeigt.
\newpage
\begin{figure}[h]
	\centering
	\includegraphics[scale=0.4]{\imagesRoot/activity_routes_menu.PNG}
	\caption
	{Routen Aktivität Menü}
\end{figure}
Die Aktivität stellt drei Optionen \emph{Reset All, Reload,  Close} über ein Menü zur Verfügung.
\begin{enumerate}
	\item\emph{Reset All}
	\newline
	Setzt alle zur Verfügung stehenden Routen zurück
	\item\emph{Reload}
	\newline
	Lädt alle zur Verfügung stehenden Routen erneut
	\item\emph{Close}
	\newline
	Öffnet einen Dialog, über den im Falle einer Bestätigung die Anwendung geschlossen wird.
\end{enumerate}
\begin{figure}[h]
	\centering
	\includegraphics[scale=0.4]{\imagesRoot/activity_routes_no_network.PNG}
	\caption
	{Keine Netzwerkverbindung}
\end{figure}
Sollte keine Netzwerkverbindung vorhanden sein so wird ein entsprechender Dialog angezeigt.
\begin{figure}[h]
	\centering
	\includegraphics[scale=0.4]{\imagesRoot/activity_routes_close_dialog.PNG}
	\caption
	{Anwendung schließen Dialog}
\end{figure}
\ \newline
Wird \emph{Yes} gedrückt wird die Anwendung geschlossen, im Falle von \emph{No} wir lediglich der \emph{Dialog} geschlossen.
\newpage
\subsubsection{Checkpoint Aktivität}
Folgend sind die Tests für die Checkpoint Aktivität angeführt.
\begin{figure}[h]
	\centering
	\includegraphics[scale=0.4]{\imagesRoot/activity_checkpoint_progress_1.PNG}
	\caption
	{Offene Route Teil 1}
\end{figure}
\ \newline
\begin{figure}[h]
	\centering
	\includegraphics[scale=0.4]{\imagesRoot/activity_checkpoint_progress_2.PNG}
	\caption
	{Offene Route Teil 2}
\end{figure}
\ \newline
Nachdem eine Route ausgewählt wurde wird auf die Checkpoint Aktivität gewechselt. Es gibt zwei Möglichkeiten die Checkpoints anzuzeigen.
\begin{enumerate}
	\item\emph{Google Map}
	\newline 
	Zeigt alle Checkpoints an wobei ein noch nicht besuchter Checkpoint rot markiert wird
	\item\emph{Checkpoint Liste}
	\newline 
	Zeigt alle besuchten Checkpoints in Listenform an wobei ein nicht besuchter Checkpoint ebenfalls rot markiert wird.
\end{enumerate}
In diesem Fall ist die Route noch nicht abgeschlossen.
\newpage
\begin{figure}[h]
	\centering
	\includegraphics[scale=0.4]{\imagesRoot/activity_checkpoint_finished_progress_menu.PNG}
	\caption
	{Optionen einer abgeschlossene oder gestarteten Route}
\end{figure}
Bei einer abgeschlossenen oder bereits gestarteten Route stehen die zwei Optionen \emph{Reset, Close} zur Verfügung.
\begin{enumerate}
	\item\emph{Reset}
	\newline
	Setzt die aktuelle Route zurück
	\item\emph{Close}
	\newline
	Öffnet einen Dialog wobei bei dessen Bestätigung die Anwendung geschlossen wird.
\end{enumerate}
\begin{figure}[h]
	\centering
	\includegraphics[scale=0.4]{\imagesRoot/activity_checkpoint_open_menu.PNG}
	\caption
	{Optionen einer offenen Route}
\end{figure}
\ \newline
Bei einer offene Route steht nur die eine Option \emph{Close} zur Verfügung.
\newpage
\begin{figure}[h]
	\centering
	\includegraphics[scale=0.4]{\imagesRoot/activity_checkpoint_wrong_secret.PNG}
	\caption
	{Frage falsch beantwortet}
\end{figure}
Sollte eine Frage falsch beantwortet worden sein so wird eine Meldung in Form eines Toast ausgegeben.
\begin{figure}[h]
	\centering
	\includegraphics[scale=0.4]{\imagesRoot/activity_checkpoint_correct_secret.PNG}
	\caption
	{Frage korrekt beantwortet}
\end{figure}
\ \newline
Sollte eine Frage korrekt beantwortet worden sein so wird eine Meldung in Form eines \emph{Toast} ausgegeben.
\ \newline
\newpage
\begin{figure}[h]
	\centering
	\includegraphics[scale=0.4]{\imagesRoot/activity_checkpoint_congratulations.PNG}
	\caption
	{Route abgeschlossen}
\end{figure}
Nachdem die letzte Frage korrekt beantwortet wurde  wird ein \emph{Dialog} mit einer Erfolgsmeldung ausgegeben. Es wird auf der Checkpoint Aktivität verblieben wobei die Ansicht sich wie bei einer abgeschlossenen Route ändert.
\newline
\begin{figure}[h]
	\centering
	\includegraphics[scale=0.4]{\imagesRoot/activity_checkpoint_close_dialog.PNG}
	\caption
	{Anwendung schließen Dialog}
\end{figure}
\ \newline
Wird \emph{Yes} gedrückt wird die Anwendung geschlossen, im Falle von \emph{No} wir lediglich der \emph{Dialog} geschlossen.
\end{document} 